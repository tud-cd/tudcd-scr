\documentclass[useseriffont=false,fontsize=cdfont]{tudcdreprt}
\usepackage[T1]{fontenc}
\usepackage[ngerman=ngerman-x-latest]{hyphsubst}

\usepackage[ngerman]{babel}
\usepackage[babel]{microtype}
\babelprovide[hyphenrules=ngerman-x-latest]{ngerman}
\usepackage{xcolor}

\usepackage{amsmath}
\usepackage{accents}
\usepackage{mathtools}
\let\openbox\undefined
\usepackage{keytheorems}
%\usepackage{showframe}

\selectcolormodel{RGB}
%\usepackage{tudcdcolor}

%\TUDCDoption{paper}{a4paper}
\TUDCDoption{usemargin}{false}
\TUDCDoption{logocolor}{blue}
%\TUDCDoption{logolang}{english}
%\TUDCDoption{logomodel}{cmyk}

\usepackage{multicol}
\usepackage{marginnote}
\usepackage{csquotes}
\usepackage{ragged2e}
\usepackage{blindtext}

\title{Die ursprüngliche Demodatei}
\subtitle{Welche die einzelnen Eigenschaften hoffentlich hat, die sie brauch}
\author{Jochen Diepelt}
\date{\today}

%\selectcolormodel{cmyk}
\sethighlightcolor{Brilliantblau}

\begin{document}

\maketitle

\tableofcontents%

\clearpage%

\RaggedRight%

\begin{multicols}{2}

\chapter{Ich bin Titel}

Ich bin ein Text, welcher etwas Struktur.

%\blindmathpaper
\end{multicols}


\begin{tudcdquotepage}[Rot2]
  \ZitatGrossgroesse\noindent\raggedright\textcolor{Brilliantblau}{\textrm{\textcdl{\textit{
    \enquote{Lasst, die ihr eintretet, alle Hoffnung fahren!}}}}}\par
\end{tudcdquotepage}


\chapter{Schriftvorstellung mit einem Kapiteltitel, welcher über zwei Spalten geht}

Ich bin ein Pangramm

\begin{multicols}{2}
Stanleys Expeditionszug quer durch Afrika wird von jedermann bewundert.

\section{Besondere Auszeichnungsbefehle}

Was ist ein Text
\verb|\textbf|
\textbf{Stanleys Expeditionszug quer durch Afrika wird von jedermann bewundert.}

\verb|\textrm|
\textrm{Stanleys Expeditionszug quer durch Afrika wird von jedermann bewundert.}

\verb|\textsf|
\textsf{Stanleys Expeditionszug quer durch Afrika wird von jedermann bewundert.}

\verb|\texttt|
\texttt{Stanleys Expeditionszug quer durch Afrika wird von jedermann bewundert.}

\verb|\textsl|
\textsl{Stanleys Expeditionszug quer durch Afrika wird von jedermann bewundert.}

\verb|\textsc|
\textsc{Stanleys Expeditionszug quer durch Afrika wird von jedermann bewundert.}

\verb|\textup|
\textup{Stanleys Expeditionszug quer durch Afrika wird von jedermann bewundert.}

\verb|\textit|
\textit{Stanleys Expeditionszug quer durch Afrika wird von jedermann bewundert.}
%


\subsection{Besondere Auszeichnungsbefehle in Kombinationen}

        Stanleys \textbf{Expeditionszug} \textrm{quer} \textsf{durch} \texttt{Afrika} \textsl{wird} \textsc{von} \textup{jedermann} \textit{bewundert.}

\textbf{Stanleys \textbf{Expeditionszug} \textrm{quer} \textsf{durch} \texttt{Afrika} \textsl{wird} \textsc{von} \textup{jedermann} \textit{bewundert.}}

\textrm{Stanleys \textbf{Expeditionszug} \textrm{quer} \textsf{durch} \texttt{Afrika} \textsl{wird} \textsc{von} \textup{jedermann} \textit{bewundert.}}

\textsf{Stanleys \textbf{Expeditionszug} \textrm{quer} \textsf{durch} \texttt{Afrika} \textsl{wird} \textsc{von} \textup{jedermann} \textit{bewundert.}}

\texttt{Stanleys \textbf{Expeditionszug} \textrm{quer} \textsf{durch} \texttt{Afrika} \textsl{wird} \textsc{von} \textup{jedermann} \textit{bewundert.}}

\textsl{Stanleys \textbf{Expeditionszug} \textrm{quer} \textsf{durch} \texttt{Afrika} \textsl{wird} \textsc{von} \textup{jedermann} \textit{bewundert.}}

\textsc{Stanleys \textbf{Expeditionszug} \textrm{quer} \textsf{durch} \texttt{Afrika} \textsl{wird} \textsc{von} \textup{jedermann} \textit{bewundert.}}

\textup{Stanleys \textbf{Expeditionszug} \textrm{quer} \textsf{durch} \texttt{Afrika} \textsl{wird} \textsc{von} \textup{jedermann} \textit{bewundert.}}

\textit{Stanleys \textbf{Expeditionszug} \textrm{quer} \textsf{durch} \texttt{Afrika} \textsl{wird} \textsc{von} \textup{jedermann} \textit{bewundert.}}


\subsection{Sind alle Glyphen vorhanden?}

Dieser Satz besteht aus acht A, sechs B, sechs C, sieben D, fünfundvierzig E, acht F, vier G, neun H,
fünfundzwanzig I, einem J, einem K, zwei L, elf M, achtundzwanzig N, einem O, einem P, einem Q, sieben R,
dreizehn S, sieben T, sieben U, fünf V, vier W, einem X, einem Y, zehn Z, einem Ä, einem Ö, vier Ü und einem ß.


\textbf{Dieser Satz besteht aus acht A, sechs B, sechs C, sieben D, fünfundvierzig E, acht F, vier G, neun H,
fünfundzwanzig I, einem J, einem K, zwei L, elf M, achtundzwanzig N, einem O, einem P, einem Q, sieben R,
dreizehn S, sieben T, sieben U, fünf V, vier W, einem X, einem Y, zehn Z, einem Ä, einem Ö, vier Ü und einem ß.}

\textrm{Dieser Satz besteht aus acht A, sechs B, sechs C, sieben D, fünfundvierzig E, acht F, vier G, neun H,
fünfundzwanzig I, einem J, einem K, zwei L, elf M, achtundzwanzig N, einem O, einem P, einem Q, sieben R,
dreizehn S, sieben T, sieben U, fünf V, vier W, einem X, einem Y, zehn Z, einem Ä, einem Ö, vier Ü und einem ß.}

\textsf{Dieser Satz besteht aus acht A, sechs B, sechs C, sieben D, fünfundvierzig E, acht F, vier G, neun H,
fünfundzwanzig I, einem J, einem K, zwei L, elf M, achtundzwanzig N, einem O, einem P, einem Q, sieben R,
dreizehn S, sieben T, sieben U, fünf V, vier W, einem X, einem Y, zehn Z, einem Ä, einem Ö, vier Ü und einem ß.}

\texttt{Dieser Satz besteht aus acht A, sechs B, sechs C, sieben D, fünfundvierzig E, acht F, vier G, neun H,
fünfundzwanzig I, einem J, einem K, zwei L, elf M, achtundzwanzig N, einem O, einem P, einem Q, sieben R,
dreizehn S, sieben T, sieben U, fünf V, vier W, einem X, einem Y, zehn Z, einem Ä, einem Ö, vier Ü und einem ß.}

\textsl{Dieser Satz besteht aus acht A, sechs B, sechs C, sieben D, fünfundvierzig E, acht F, vier G, neun H,
fünfundzwanzig I, einem J, einem K, zwei L, elf M, achtundzwanzig N, einem O, einem P, einem Q, sieben R,
dreizehn S, sieben T, sieben U, fünf V, vier W, einem X, einem Y, zehn Z, einem Ä, einem Ö, vier Ü und einem ß.}

\textsc{Dieser Satz besteht aus acht A, sechs B, sechs C, sieben D, fünfundvierzig E, acht F, vier G, neun H,
fünfundzwanzig I, einem J, einem K, zwei L, elf M, achtundzwanzig N, einem O, einem P, einem Q, sieben R,
dreizehn S, sieben T, sieben U, fünf V, vier W, einem X, einem Y, zehn Z, einem Ä, einem Ö, vier Ü und einem ß.}

\textup{Dieser Satz besteht aus acht A, sechs B, sechs C, sieben D, fünfundvierzig E, acht F, vier G, neun H,
fünfundzwanzig I, einem J, einem K, zwei L, elf M, achtundzwanzig N, einem O, einem P, einem Q, sieben R,
dreizehn S, sieben T, sieben U, fünf V, vier W, einem X, einem Y, zehn Z, einem Ä, einem Ö, vier Ü und einem ß.}

\textit{Dieser Satz besteht aus acht A, sechs B, sechs C, sieben D, fünfundvierzig E, acht F, vier G, neun H,
fünfundzwanzig I, einem J, einem K, zwei L, elf M, achtundzwanzig N, einem O, einem P, einem Q, sieben R,
dreizehn S, sieben T, sieben U, fünf V, vier W, einem X, einem Y, zehn Z, einem Ä, einem Ö, vier Ü und einem ß.}

\textsw{Dieser Satz besteht aus acht A, sechs B, sechs C, sieben D, fünfundvierzig E, acht F, vier G, neun H,
fünfundzwanzig I, einem J, einem K, zwei L, elf M, achtundzwanzig N, einem O, einem P, einem Q, sieben R,
dreizehn S, sieben T, sieben U, fünf V, vier W, einem X, einem Y, zehn Z, einem Ä, einem Ö, vier Ü und einem ß.}

%\text{Dieser Satz besteht aus \textbf{acht A}, \textit{sechs B}, \textsc{sechs C}, \texttt{sieben D}, \textrm{fünfundvierzig E}, acht F, vier G, neun H,
%fünfundzwanzig I, einem J, einem K, zwei L, elf M, achtundzwanzig N, einem O, einem P, einem Q, sieben R,
%dreizehn S, sieben T, sieben U, fünf V, vier W, einem X, einem Y, zehn Z, einem Ä, einem Ö, vier Ü und einem ß.}


{\tiny Ich bin \verb|\tiny| mit \par}

{\scriptsize Ich bin \verb|\scriptsize| mit\par}

{\footnotesize Ich bin \verb|\footnotesize| mit\par}

{\small Ich bin \verb|\small| mit\par}

{\normalsize Ich bin \verb|\normalsize| mit\par}

{\large Ich bin \verb|\large| mit\par}

{\Large Ich bin \verb|\Large| mit\par}

{\LARGE Ich bin \verb|\LARGE| mit\par}

{\huge Ich bin \verb|\huge| mit\par}
\end{multicols}




\begin{tudcdquotepage}[Gruen2]
  \ZitatGrossgroesse\noindent\raggedright\textcolor{Brilliantblau}{\textrm{\textcdl{\textit{
    \enquote{Das Beste an der \\ TUD ist, dass der Schwerpunkt auf der Forschung liegt.
    Es ist ganz wichtig, dass die Dozent:innen uns nicht nur von der gängigen Praxis erzählen,
    sondern auch einen Ausblick für die Zukunft geben.}}}}}\par
\end{tudcdquotepage}

\newpage
\nopagecolor
{
\Large
\begin{testcolors}[RGB,cmyk]

    \testcolor{Brilliantblau}
    \testcolor{Dunkelblau}
    \testcolor{Blau1}
    \testcolor{Blau2}
    \testcolor{Violett1}
    \testcolor{Violett2}
    \testcolor{Magenta1}
    \testcolor{Magenta2}
    \testcolor{Rot1}
    \testcolor{Rot2}
    \testcolor{Orange1}
    \testcolor{Orange2}
    \testcolor{Gelb1}
    \testcolor{Gelb2}
    \testcolor{Oliv1}
    \testcolor{Oliv2}
    \testcolor{Gruen1}
    \testcolor{Gruen2}
    \testcolor{Tuerkis1}
    \testcolor{Tuerkis2}
\end{testcolors}

}


\end{document}