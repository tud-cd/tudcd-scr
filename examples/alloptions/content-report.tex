% arara: pdflatex: { files: [ "main.tex" ]}

\begin{document}

\maketitle%
%
\RaggedRight%

\chapter{Ich bin die Kapitelüberschrift}

\begin{multicols}{2}

Bei Kapiteln werden neue Seiten begonnen, daher ist es im dem Zusammenhang naheliegend die Aufgabenbearbeitungen mit Kapiteln zu trennen.
Kapitelüberschriften landen im Inhaltsverzeichnis.


\section{Grundlagen für Benutzer von LaTeX} % Übrigens gibt es auch Kommentare, die landen dann nicht im Enddokument.

\input{cont.tex}

\end{multicols}
\iffalse
Bei Tabellen gibt es andere Vorgaben, in Deutschland sind diese mit einer Überschrift anstelle einer Unterschrift auszustatten.

\begin{table}[H]
    \centering
    \captionabove{Ich bin die Tabellenüberschrift}
    \begin{tabular}{|c|c|}
        \hline
        Ich bin eine Zelle           & ab in die LibreOfficeCalc Tabelle \\
        \cline{1-1}
        $c=3$ im Mathematischen Satz & \textit{Ich bin sogar schräggestellt} \\
        \hline
    \end{tabular}
    \label{tab:MUSTER-Tabellenreferenz}
\end{table}

Auf Tabellen kann mit dem gleichen Mechanismus \ref{tab:MUSTER-Tabellenreferenz} verwiesen werden.

\section{Zitieren und Quellenangaben}
\fi


\end{document}