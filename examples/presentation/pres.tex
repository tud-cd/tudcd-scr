\documentclass{beamer}

\usetheme[englishlogo]{tudcd}

\usepackage[utf8]{inputenc}
\usepackage[T1]{fontenc}
\usepackage[ngerman]{babel}
\usepackage{amsmath,amssymb,amsthm}
\usepackage{lipsum}
\usepackage{calc}

\iffalse
\usepackage{pgfpages}
\pgfpagesuselayout{2 on 1}[a4paper,border shrink=5mm]
\fi
\usepackage{tikz}
\iffalse
\AddToHook{shipout/foreground}{
\begin{tikzpicture}[remember picture,overlay]
    \coordinate (top) at (current page.north west);
    \coordinate (bottom) at (current page.south west);
    \fill[magenta,opacity=0.2] (top) -- +(\paperwidth,0mm) -- +(\paperwidth,-5.2mm) -- +(0,-5.2mm);
    \fill[magenta,opacity=0.2] (top) ++(8mm,-5.2mm) rectangle +(\paperwidth-16mm,-8.75mm);
    \fill[magenta,opacity=0.4] (top) ++(8mm,-5.2mm-8.75mm) rectangle +(\paperwidth-16mm,-8.75mm);
    \fill[magenta,opacity=0.2] (top) -- +(\paperwidth,0mm) -- +(\paperwidth,-5.2mm) -- +(0,-5.2mm);

    \fill[magenta,opacity=0.2] (bottom) -- +(\paperwidth,0mm) -- +(\paperwidth,5.2mm) -- +(0,5.2mm);
    \fill[magenta,opacity=0.5] (bottom) ++(8mm,5.2mm) rectangle +(6*9.8mm+5*2.4mm,2.1mm);% -- +(6*9.8mm+5*2.4mm,2.1mm) -- +(0,2.1mm);
    \fill[magenta,opacity=0.5] (bottom) ++(8mm+6*9.8mm+6*2.4mm,5.2mm) rectangle +(6*9.8mm+5*2.4mm,5.2mm);
    \fill[magenta,opacity=0.5] (bottom) ++(8mm,10.4mm) rectangle +(12*9.8mm+11*2.4mm,2.9mm);

    \fill[magenta,opacity=0.2] (top) ++(8mm,-5.2mm-2*8.75mm) rectangle +(9.8mm,-\paperheight+5.2mm+2*8.75mm+5.2mm);
    \foreach \p in {1,2,...,11}{%
      \fill[magenta,opacity=0.2] (top) ++(8mm+\p*2.4mm+\p*9.8mm,-5.2mm-2*8.75mm) rectangle +(9.8mm,-\paperheight+5.2mm+2*8.75mm+5.2mm);
    }
    %\node[red,scale=10,opacity=0.2] at (current page.north west) {Draft};
  \end{tikzpicture}
}
\fi

\title[Kurztitel der Präsentation]{Titel -- Inhalt des Vortrags, \\
Noto Sans Semibold 19 pt, \\
Zeilenanstand 21 pt}
\author[Vortragende]{Yoshi Diepelt}
\institute{Institut für Numerik | Institut für Mechanik und Flächentragwerke}
%\group[Short Group]{Group}
\context[Short Context]{Context}
\location[Short Location]{Tokyo}
\date{\today}
%
\setbeamertemplate{title page/colormode}[blue]

\begin{document}

\maketitle

\setbeamertemplate{title page/colormode}[white]

\maketitle

\begin{frame}
  \frametitle{Ich bin ein Titel}
    \framesubtitle{Ich bin ein Untertitel}

    Ich bin ein Inhalt auf der Folie
\end{frame}


\begin{frame}
  \frametitle{Die ist ein \textbackslash frametitle Frametitel}
  \framesubtitle{Hier steht der \textbackslash framesubtitle}
  \lipsum[1][1-4]
  \begin{itemize}
    \item Erstes item
    \item Zweites item mit inline Formel: $E = mc^2$
    \item Drittes item mit abgesetzter Formel:
      \[
        \int_a^b f(x)\,dx = F(b) - F(a).
      \]
  \end{itemize}
\end{frame}

\begin{frame}
  \frametitle{Die ist ein zweiter Frame}
  \framesubtitle{Noch ein Untertitel}
  \lipsum[2][1-3]\footnote{Eine Fußnote mit Nummerierung}

  \begin{enumerate}
    \item Erstes enumerated item
    \item Zweites enumerated item
    \item Drittes enumerated item
  \end{enumerate}

  \rule{\textwidth}{1em}

  %\blankfootnote{Eine Fußnote ohne Nummerierung}
\end{frame}

\begin{frame}
  \frametitle{Ein Frame mit Boxen}
  \lipsum[1][5-6]

  \onslide<2->{%
  \begin{columns}[t,onlytextwidth]
    \begin{column}{.5\textwidth-1em}
    \begin{block}{Blocktitel Links}
      \lipsum[1][7-8]
    \end{block}
    \end{column}
    \begin{column}{.5\textwidth-1em}
    \begin{block}{Rechts}
      \lipsum[1][9-10]
    \end{block}
    \end{column}
  \end{columns}}
\end{frame}


\begin{frame}
  \frametitle{Frame mit Beispiel und Alertbox}

  In diesem Text gibt es eine \alert{hervorgehobenes} Wort und spezielle Boxen:
  \languagename

  \begin{exampleblock}{Beispiel}
    \lipsum[1][11-12]
  \end{exampleblock}
  \begin{alertblock}{Alertbox}
    \lipsum[1][13-14]
  \end{alertblock}
\end{frame}

\begin{frame}
  \frametitle{Frame mit Theorem und Definition}

  \begin{theorem}[Pythagoras]
    Für ein rechtwinkliges Dreieck mit den Katheten $a$ und $b$ und der Hypotenuse $c$ gilt:
    \[
      a^2 + b^2 = c^2.
    \]
  \end{theorem}

  \begin{definition}[Abgeschlossene Menge]
    Eine Menge $M$ heißt \emph{abgeschlossen}, wenn für alle Folgen $(a_n)_{n\in\mathbb{N}}$ in $M$ mit $a_n\to a$ auch $a\in M$ gilt.
  \end{definition}

\end{frame}


\begin{frame}

  \begin{columns}[t]
    \begin{column}{0.4\textwidth}
  \begin{figure}
\includegraphics[height=0.4\textheight]{example-image-a}
\caption{This caption is placed below the figure.}
\end{figure}
    \end{column}
    \begin{column}{0.4\textwidth}
  \begin{figure}
\includegraphics[height=0.4\textheight]{example-image-b}
\caption{This caption is placed below the figure.}
\end{figure}
    \end{column}
  \end{columns}


\end{frame}

\begin{frame}
\begin{itemize}
  \item{\textcdul{Ich bin ein Test Ultra Light}}%
  \item{\textcdel{Ich bin ein Test Extra Light}}%
  \item{\textcdl{Ich bin ein Test Light}}%
  \item{\textcdsl{Ich bin ein Test Semi Light}}%
  \item{\textcdm{Ich bin ein Test Medium}}%
  \item{\textcdsb{Ich bin ein Test Semi bold}}%
  \item{\textcdb{Ich bin ein Test Bold}}%
  \item{\textcdeb{Ich bin ein Test Extra Bold}}%
  \item{\textcdub{Ich bin ein Test Ultra Bold}}%
\end{itemize}
\end{frame}

\begin{frame}
  \textcolor{blue}{\rule{\textwidth}{\dimexpr(90mm-5.2mm-5.2mm-5.2mm-2.9mm)}}
\end{frame}

\end{document}