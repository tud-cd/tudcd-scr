\documentclass[
  logo={
    color=blue,
    language=de,
    colormodel=rgb
  },
  fontsize=11tudpt,
  usemargin,
  oneside
]{tudcdreprt}
\usepackage[T1]{fontenc}
\usepackage[utf8]{inputenc}
\usepackage[ngerman=ngerman-x-latest]{hyphsubst}

\usepackage[ngerman]{babel}
\usepackage[babel]{microtype}
\babelprovide[hyphenrules=ngerman-x-latest]{ngerman}

\usepackage{biblatex}
\addbibresource{handbook.bib}
\addbibresource{source.bib}
\usepackage{csquotes}
\usepackage{hvlogos}

\usepackage{siunitx}
\usepackage{float}
\usepackage{tabularray}
\SetTblrInner{expand=\tudcdGridColumnwidth}
\usepackage{listings}
\lstset{
  language={[LaTeX]Tex},
  basicstyle=\small\ttfamily,
  literate=%
  {Ö}{{\"O}}1
  {Ä}{{\"A}}1
  {Ü}{{\"U}}1
  {ß}{{\ss}}1
  {ü}{{\"u}}1
  {ä}{{\"a}}1
  {ö}{{\"o}}1
  {~}{{\textasciitilde}}1,
  numberstyle=\footnotesize,
  frame=single,
  rulecolor=\SelectTUDCDHighlightColor,
  backgroundcolor=\SelectTUDCDTertiaryColor
%  numbers=left,
}

\usepackage{hyperref}

\author{Jochen Diepelt}
\title{TUDCD-Script: Offizielle Klassen und Pakete im Stil der TU Dresden}
\subject{Handbuch}
\date{2026.01.30}

\TUDCDsetup{
  titlepage={
    highlight-color=Weiss,
    background-color=Brilliantblau,
    logo-color=white
  },
  color={
    highlight-color=Brilliantblau,
    tertiary-color=Blau2%
  }
}

\begin{document}

\maketitle%

%%%%%%%%%%

\tableofcontents

%%%%%%%%%%

\chapter{Einführung}

Mit der Erneuerung des Corporate Designs der Technischen Universität Dresden wurden auch \LaTeX{} Klassen
mit in Auftrag gegeben.
Dabei wurde sich dazu entschieden, die \enquote{inoffiziellen Klassen} von Falk Hanisch~\cite{ctan-tudscr}
abzulösen und eine neue Paketsammlung zu erstellen.
Aus Archivierungsgründen bleibt jedoch die Paketsammlung \texttt{tudscr} auf \CTAN{} als solche erhalten,
und daher wurde der Name \textbf{\texttt{tudcdscr}} für die neue Paketsammlung gewählt.

\section{Allgemeine Einstellungen}

Die grundlegenden Einstellungen von Optionen erfolgt über das Makro \lstinline|\TUDCDsetup|,
welches die Schlüssel-Wert-Paare der Klassenoptionen entgegennimmt.

\begin{lstlisting}
\TUDCDsetup{
  <Schlüssel>=<Wert>
}
\end{lstlisting}

Die Optionen unterteilen sich dabei in drei Kategorien:
\begin{enumerate}
  \item globale Optionen, welche in \lstinline|\documentclass| geladen werden müssen,
  \item späte Optionen, welche in der Dokumentenpräambel über \lstinline|\TUDCDsetup| geladen werden und
  \item einfache Optionen, welche zu jeder Zeit im Dokument über \lstinline|\TUDCDsetup| geladen werden können.
\end{enumerate}
Einfache und späte Optionen können ebenfalls in \lstinline|\documentclass| eingestellt werden,
während einfache Optionen auch in der Dokumentenpräambel geladen werden können.

\section{Einstellungen des Logos}

Das Logo der TUD hat 3 Merkmale, welche unabhängig voneinander eingestellt werden können,
\begin{itemize}
  \item die Logosprache, in Deutsch und Englisch,
  \item die Logofarbe, in Blau, Weiß und Schwarz, sowie
  \item das Logofarbmodel, in CMYK und in RGB.
\end{itemize}
Diese Merkmale werden über die einfachen Optionen
\begin{lstlisting}
\TUDCDsetup{
  logo/language=<Wert>,
  logo/color=<Wert>,
  logo/colormodel=<Wert>
}
\end{lstlisting}
eingestellt.

\begin{description}
  \item[\texttt{logo/language}] besitzt die zulässigen Werte \texttt{de}, \texttt{en} und \texttt{auto}
  \item[\texttt{logo/color}] besitzt die zulässigen Werte \texttt{white}, \texttt{black} und \texttt{blue}
  \item[\texttt{logo/model}] besitzt die zulässigen Werte \texttt{cmyk}, \texttt{rgb} und \texttt{auto}
\end{description}
Bei der Einstellung \texttt{auto} versucht die Klasse zu einem geeigneten Zeitpunkt die Option aus anderen Informationen
zu setzen, dies muss jedoch zwischen Versionen nicht unbedingt stabil bleiben.
Zu bevorzugen sind definitiv die festen Werte.

Die Einstellungen können ebenfalls über \texttt{color} gesetzt werden
\begin{lstlisting}
\TUDCDsetup{
  logo={
    language=<Wert>,
    color=<Wert>,
    colormodel=<Wert>
  }
}
\end{lstlisting}

Mit den Logoeinstellungen kann anschließend das Logo mittels \lstinline|\TypesetTUDLogo| gesetzt werden.
\begin{lstlisting}
\TypesetTUDLogo{<Dimension>}
\end{lstlisting}

\textbf{ACHTUNG:} Das Logo besitzt eine Unterlänge, weswegen der Shape des Logos immer \emph{kleiner} sein wird,
als die angegeben Größe. Die Dimension muss mit dem Faktor \num{1.13} multipliziert werden, damit der Shape die angegebene Größe hat.


\section{Einstellungen der Farben}

Das Corporate Design sieht für Farben eine festgelegte Farbpalette vor.
Gemäß Corporate Design Leitfaden dürfen maximal zwei unterschiedliche
sekundäre Farben verwendet werden, allerdings wird hierbei die Schwarze Schriftfarbe
nicht gezählt.

TUDCD-Script bietet daher drei einstellbare Farben als einfache Optionen an:
\begin{lstlisting}
\TUDCDsetup{
  color/highlight-color=<Farbe>,
  color/secondary-color=<Farbe>,
  color/tertiary-color=<Farbe>
}
\end{lstlisting}
welche jeweils mit
\begin{lstlisting}
\SelectTUDCDHighlightColor
\SelectTUDCDSecondaryColor
\SelectTUDCDTertiaryColor
\end{lstlisting}
ausgewählt werden können.

\textbf{ACHTUNG:} Die \lstinline|\SelectTUDCD...|-Befehle wählen die Farbe in der momentanen Gruppe aus!
Um also die Farbauswahl zu begrenzen ist entweder
\begin{lstlisting}
{
\SelectTUDCDHighlightColor
Hier ein Text
}
\end{lstlisting}
oder
\begin{lstlisting}
\begingroup
\SelectTUDCDHighlightColor
Hier ein Text
\endgroup
\end{lstlisting}
zu benutzten.

\textbf{ANMERKUNG:} Die Makros \lstinline|\textcolor|,\lstinline|\mathcolor| und \lstinline|\color| des Pakets \texttt{xcolor}~\cite{ctan-xcolor},
können bisher nicht mit den Auszeichnungsfarben verwendet werden. Eine Möglichkeit für Metafarben, welche in diesen Befehlen
verwendet werden kann, wird in Kürze bereitgestellt.

\subsection{Der Farbenkatalog des Corporate Designs}

Die Farben des Corporate Design werden in Tabelle~\ref{tab:colorcatalogue} dargestellt.
Dabei wurde die zulässige Schriftfarbe zu einem Hintergrund mit dargestellt.
Auf den Sekundärfarben (mit dem Namen \texttt{Farbe2})
darf ebenfalls mit dem Brilliantblau geschrieben werden.
Dabei sollte beachtet werden, dass die Kontraste mindestens dem Grad AA
der WCAG 2.2 Norm erfüllen. % TODO: ORDENTLICHE QUELLE HINZUFÜGEN!
Eine Übersicht über die Kontraste der Farben finden sie unter diesem
\href{https://contrast-grid.eightshapes.com/?version=1.1.0&background-colors=&foreground-colors=%2300005A%2C%20Dunkelblau%0D%0A%2300008C%2C%20Prim%C3%A4rblau%0D%0A%232F57B2%2C%20Blau%201%0D%0A%237369BE%2C%20Violett%201%0D%0A%23BC1589%2C%20Pink%201%0D%0A%23D20F41%2C%20Rot%201%0D%0A%23C85000%2C%20Orange%201%0D%0A%23FFC700%2C%20Gelb%201%0D%0A%23767A23%2C%20Oliv%201%0D%0A%23007D4B%2C%20Gr%C3%BCn%201%0D%0A%230A777F%2C%20Petrol%201%0D%0A%2397C6FF%2C%20Blau%202%0D%0A%23C8C8FF%2C%20Violett%202%0D%0A%23FFB9FF%2C%20Pink%202%0D%0A%23FFAAA5%2C%20Rot%202%0D%0A%23FFBE78%2C%20Orange%202%0D%0A%23FFE483%2C%20Gelb%202%0D%0A%23D2DC46%2C%20Oliv%202%0D%0A%238CE6AA%2C%20Gr%C3%BCn%202%0D%0A%238CE6D7%2C%20Petrol%202%0D%0A%23000000%2C%20Schwarz%0D%0A%23323F4B%2C%20Grau%20100%0D%0A%23566371%2C%20Grau%2080%0D%0A%237D8894%2C%20Grau%2060%0D%0A%23A5AEB8%2C%20Grau%2040%0D%0A%23D0D5DC%2C%20Grau%2020%0D%0A%23E7E9ED%2C%20Grau%2010%0D%0A%23FFFFFF%2C%20Wei%C3%9F&es-color-form__tile-size=compact&es-color-form__show-contrast=aaa&es-color-form__show-contrast=aa&es-color-form__show-contrast=aa18&es-color-form__show-contrast=dnp}{Link}.


\begin{table}[h]
  \centering
  \label{tab:colorcatalogue}
  \captionabove{Farbenkatalog der TU Dresden}
  \begin{tabular}{
    @{}
    p{\tudcdGridColumnwidth}
    @{\hspace{\tudcdGridColumnsep}}
    p{\tudcdGridColumnwidth}
    @{\hspace{\tudcdGridColumnsep}}
    p{\tudcdGridColumnwidth}
    @{\hspace{\tudcdGridColumnsep}}
    p{\tudcdGridColumnwidth}@{}}
    \colorbox{Brilliantblau}{\parbox{\tudcdGridColumnwidth}{\textcolor{Weiss}{Brilliantblau} \\}} &
    \colorbox{Blau1}{\parbox{\tudcdGridColumnwidth}{\textcolor{Weiss}{Blau1} \\ }} &
    \colorbox{Violett1}{\parbox{\tudcdGridColumnwidth}{\textcolor{Weiss}{Violett1} \\}} &
    \colorbox{Magenta1}{\parbox{\tudcdGridColumnwidth}{\textcolor{Weiss}{Magenta1} \\}} \\
    %
    \colorbox{Dunkelblau}{\parbox{\tudcdGridColumnwidth}{\textcolor{Weiss}{Dunkelblau} \\ }} &
    \colorbox{Blau2}{\parbox{\tudcdGridColumnwidth}{\textcolor{Schwarz}{Blau2} \\ }} &
    \colorbox{Violett2}{\parbox{\tudcdGridColumnwidth}{\textcolor{Schwarz}{Violett2} \\}} &
    \colorbox{Magenta2}{\parbox{\tudcdGridColumnwidth}{\textcolor{Schwarz}{Magenta2}  \\}} \\[\tudcdGridColumnsep]
    % Platzlassen
    \colorbox{Rot1}{\parbox{\tudcdGridColumnwidth}{\textcolor{Weiss}{Rot1} \\}} &
    \colorbox{Orange1}{\parbox{\tudcdGridColumnwidth}{\textcolor{Weiss}{Orange1} \\ }} &
    \colorbox{Gelb1}{\parbox{\tudcdGridColumnwidth}{\textcolor{Schwarz}{Gelb1} \\}} &
    \colorbox{Oliv1}{\parbox{\tudcdGridColumnwidth}{\textcolor{Weiss}{Oliv1} \\}} \\
    %
    \colorbox{Rot2}{\parbox{\tudcdGridColumnwidth}{\textcolor{Schwarz}{Rot2} \\ }} &
    \colorbox{Orange2}{\parbox{\tudcdGridColumnwidth}{\textcolor{Schwarz}{Orange2} \\ }} &
    \colorbox{Gelb2}{\parbox{\tudcdGridColumnwidth}{\textcolor{Schwarz}{Gelb2} \\}} &
    \colorbox{Oliv2}{\parbox{\tudcdGridColumnwidth}{\textcolor{Schwarz}{Oliv2}  \\}} \\[\tudcdGridColumnsep]
    % Platzlassen
    \colorbox{Gruen1}{\parbox{\tudcdGridColumnwidth}{\textcolor{Weiss}{Gruen1} \\}} &
    \colorbox{Tuerkis1}{\parbox{\tudcdGridColumnwidth}{\textcolor{Weiss}{Tuerkis1} \\ }} &
    \colorbox{Schwarz}{\parbox{\tudcdGridColumnwidth}{\textcolor{Weiss}{Schwarz} \\}} &
    \colorbox{Grau100}{\parbox{\tudcdGridColumnwidth}{\textcolor{Weiss}{Grau100} \\}} \\
    %
    \colorbox{Gruen2}{\parbox{\tudcdGridColumnwidth}{\textcolor{Schwarz}{Gruen2} \\ }} &
    \colorbox{Tuerkis2}{\parbox{\tudcdGridColumnwidth}{\textcolor{Schwarz}{Tuerkis2} \\ }} &
    \colorbox{Weiss}{\parbox{\tudcdGridColumnwidth}{\textcolor{Schwarz}{Weiss} \\}} &
    \colorbox{Grau80}{\parbox{\tudcdGridColumnwidth}{\textcolor{Weiss}{Grau80}  \\}} \\[\tudcdGridColumnsep]

    \colorbox{Grau60}{\parbox{\tudcdGridColumnwidth}{\textcolor{Weiss}{Grau60} \\ }} &
    \colorbox{Grau40}{\parbox{\tudcdGridColumnwidth}{\textcolor{Schwarz}{Grau40} \\ }} &
    \colorbox{Grau20}{\parbox{\tudcdGridColumnwidth}{\textcolor{Schwarz}{Grau20} \\}} &
    \colorbox{Grau10}{\parbox{\tudcdGridColumnwidth}{\textcolor{Schwarz}{Grau10}  \\}} \\
  \end{tabular}
\end{table}




\subsection{Die Hausschrift der TU Dresden}

Die Hausschrift der TUD ist die Noto Sans, im Schriftschnitt Medium.
Da für Fließtexte eine Serifenschrift unter Umständen einen besseren
Lesefluss ermöglicht, wird die späte Option \texttt{useseriffont} bereitgestellt
\begin{lstlisting}
\documentclass{<TUDCD-Klasse>}

\TUDCDsetup{
  useseriffont
}
% oder
\TUDCDsetup{
  useseriffont=<Wahrheitswert>
}
% ...
\begin{document}
\end{lstlisting}
Diese Option schaltet auf die Serifenschrift um.
Dabei bleibt für Überschriften die serifenlose Schrift erhalten.


\TUDCDsetup{
  color={
    highlight-color=Gruen1,
    tertiary-color=Gruen2
  }
}

\chapter{Dokumente im Stil der TU Dresden}
  Hauptklassen \texttt{tudcdartcl} und \texttt{tudcdreprt}

Zum Zeitpunkt des Schreibens sind die beiden Hauptklassen
\texttt{tudcdartcl} und \texttt{tudcdreprt}.

\section{Einstellungen}

wird die globale Option \texttt{usemargin} bereitgestellt.
\begin{lstlisting}
\documentclass[usemargin]{<Klasse>}
% Oder
\documentclass[usemargin=<Wahrheitswert>]{<Klasse>}
\end{lstlisting}


\subsection{Wieder anderer Test}

\subsubsection{Noch ein Test}

\TUDCDsetup{
  color={
    highlight-color=Brilliantblau
  }
}
\printbibliography%

\end{document}
