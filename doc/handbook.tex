\documentclass[
  logo={
    color=blue,
    language=de,
    colormodel=rgb
  },
  fontsize=11tudpt,
  usemargin,
  oneside
]{tudcdreprt}
\usepackage[T1]{fontenc}
\usepackage[utf8]{inputenc}
\usepackage[ngerman=ngerman-x-latest]{hyphsubst}

\usepackage[ngerman]{babel}
\usepackage[babel]{microtype}
\babelprovide[hyphenrules=ngerman-x-latest]{ngerman}

\usepackage{biblatex}
\addbibresource{handbook.bib}
\addbibresource{source.bib}
\usepackage{csquotes}
\usepackage{hvlogos}
\usepackage{pdfpages}
\usepackage{subcaption}
\usepackage{siunitx}
\usepackage{attachfile}
\DeclareSIUnit{\texpt}{pt}
\DeclareSIUnit{\texbp}{bt}
\DeclareSIUnit{\texsp}{sp}
\DeclareSIQualifier{\word}{\text{Word}}
\DeclareSIQualifier{\tex}{\text{\TeX}}
\DeclareSIQualifier{\adobe}{\text{Adobe}}
\DeclareSIQualifier{\tud}{\text{TUD}}


\usepackage{float}
\usepackage{tabularray}
\SetTblrInner{expand=\tudcdGridColumnwidth}
\usepackage{listings}
\lstset{
  language={[LaTeX]Tex},
  basicstyle=\small\ttfamily,
  literate=%
  {Ö}{{\"O}}1
  {Ä}{{\"A}}1
  {Ü}{{\"U}}1
  {ß}{{\ss}}1
  {ü}{{\"u}}1
  {ä}{{\"a}}1
  {ö}{{\"o}}1
  {~}{{\textasciitilde}}1,
  numberstyle=\footnotesize,
  frame=single,
  breaklines,
  rulecolor=\SelectTUDCDHighlightColor,
  backgroundcolor=\SelectTUDCDTertiaryColor
%  numbers=left,
}

\usepackage{hyperref}

\author{Jochen Diepelt}
\title{TUDCD-Script: Offizielle Klassen und Pakete im Stil der TU Dresden}
\subject{Handbuch}
\date{2026.02.02}

\TUDCDsetup{
  titlepage={
    highlight-color=Weiss,
    background-color=Brilliantblau,
    logo-color=white
  },
  color={
    highlight-color=Brilliantblau,
    tertiary-color=Blau2%
  }
}
\attachfilesetup{
  color=0.823529412 0.058823529 0.254901961
}

\begin{document}

\maketitle%

%%%%%%%%%%

\tableofcontents

%%%%%%%%%%

\chapter{Einführung}

Mit der Erneuerung des Corporate Designs der Technischen Universität Dresden wurden auch \LaTeX{} Klassen
mit in Auftrag gegeben.
Dabei wurde sich dazu entschieden, die \enquote{inoffiziellen Klassen} von Falk Hanisch~\cite{ctan-tudscr}
abzulösen und eine neue Paketsammlung zu erstellen.
Aus Archivierungsgründen bleibt jedoch die Paketsammlung \texttt{tudscr} auf \CTAN{} als solche erhalten,
und daher wurde der Name \textbf{\texttt{tudcdscr}} für die neue Paketsammlung gewählt.

\section{Allgemeine Einstellungen}

Die grundlegenden Einstellungen von Optionen erfolgt über das Makro \lstinline|\TUDCDsetup|,
welches die Schlüssel-Wert-Paare der Klassenoptionen entgegennimmt.

\begin{lstlisting}
\TUDCDsetup{
  <Schlüssel>=<Wert>
}
\end{lstlisting}

Die Optionen unterteilen sich dabei in drei Kategorien:
\begin{enumerate}
  \item globale Optionen, welche in \lstinline|\documentclass| geladen werden müssen,
  \item späte Optionen, welche in der Dokumentenpräambel über \lstinline|\TUDCDsetup| geladen werden und
  \item einfache Optionen, welche zu jeder Zeit im Dokument über \lstinline|\TUDCDsetup| geladen werden können.
\end{enumerate}
Einfache und späte Optionen können ebenfalls in \lstinline|\documentclass| eingestellt werden,
während einfache Optionen auch in der Dokumentenpräambel geladen werden können.

\section{Einstellungen des Logos}

Das Logo der TUD hat 3 Merkmale, welche unabhängig voneinander eingestellt werden können,
\begin{itemize}
  \item die Logosprache, in Deutsch und Englisch,
  \item die Logofarbe, in Blau, Weiß und Schwarz, sowie
  \item das Logofarbmodel, in CMYK und in RGB.
\end{itemize}
Diese Merkmale werden über die einfachen Optionen
\begin{lstlisting}
\TUDCDsetup{
  logo/language=<Wert>,
  logo/color=<Wert>,
  logo/colormodel=<Wert>
}
\end{lstlisting}
eingestellt.

\begin{description}
  \item[\texttt{logo/language}] besitzt die zulässigen Werte \texttt{de}, \texttt{en} und \texttt{auto}
  \item[\texttt{logo/color}] besitzt die zulässigen Werte \texttt{white}, \texttt{black} und \texttt{blue}
  \item[\texttt{logo/model}] besitzt die zulässigen Werte \texttt{cmyk}, \texttt{rgb} und \texttt{auto}
\end{description}
Bei der Einstellung \texttt{auto} versucht die Klasse zu einem geeigneten Zeitpunkt die Option aus anderen Informationen
zu setzen, dies muss jedoch zwischen Versionen nicht unbedingt stabil bleiben.
Zu bevorzugen sind definitiv die festen Werte.

Die Einstellungen können ebenfalls über \texttt{color} gesetzt werden
\begin{lstlisting}
\TUDCDsetup{
  logo={
    language=<Wert>,
    color=<Wert>,
    colormodel=<Wert>
  }
}
\end{lstlisting}

Mit den Logoeinstellungen kann anschließend das Logo mittels \lstinline|\TypesetTUDLogo| gesetzt werden.
\begin{lstlisting}
\TypesetTUDLogo{<Dimension>}
\end{lstlisting}

\textbf{ACHTUNG:} Das Logo besitzt eine Unterlänge, weswegen der Shape des Logos immer \emph{kleiner} sein wird,
als die angegeben Größe. Die Dimension muss mit dem Faktor \num{1.13} multipliziert werden, damit der Shape die angegebene Größe hat.


\section{Einstellungen der Farben}

Das Corporate Design sieht für Farben eine festgelegte Farbpalette vor.
Gemäß Corporate Design Leitfaden dürfen maximal zwei unterschiedliche
sekundäre Farben verwendet werden, allerdings wird hierbei die Schwarze Schriftfarbe
nicht gezählt.

TUDCD-Script bietet daher drei einstellbare Farben als einfache Optionen an:
\begin{lstlisting}
\TUDCDsetup{
  color/highlight-color=<Farbe>,
  color/secondary-color=<Farbe>,
  color/tertiary-color=<Farbe>
}
\end{lstlisting}
welche jeweils mit
\begin{lstlisting}
\SelectTUDCDHighlightColor
\SelectTUDCDSecondaryColor
\SelectTUDCDTertiaryColor
\end{lstlisting}
ausgewählt werden können.

\textbf{ACHTUNG:} Die \lstinline|\SelectTUDCD...|-Befehle wählen die Farbe in der momentanen Gruppe aus!
Um also die Farbauswahl zu begrenzen ist entweder
\begin{lstlisting}
{
\SelectTUDCDHighlightColor
Hier ein Text
}
\end{lstlisting}
oder
\begin{lstlisting}
\begingroup
\SelectTUDCDHighlightColor
Hier ein Text
\endgroup
\end{lstlisting}
zu benutzten.

\textbf{ANMERKUNG:} Die Makros \lstinline|\textcolor|,\lstinline|\mathcolor| und \lstinline|\color| des Pakets \texttt{xcolor}~\cite{ctan-xcolor},
können bisher nicht mit den Auszeichnungsfarben verwendet werden. Eine Möglichkeit für Metafarben, welche in diesen Befehlen
verwendet werden kann, wird in Kürze bereitgestellt.

\subsection{Der Farbenkatalog des Corporate Designs}

Die Farben des Corporate Design werden in Tabelle~\ref{tab:colorcatalogue} dargestellt.
Dabei wurde die zulässige Schriftfarbe zu einem Hintergrund mit dargestellt.
Auf den Sekundärfarben (mit dem Namen \texttt{Farbe2})
darf ebenfalls mit dem Brilliantblau geschrieben werden.
Dabei sollte beachtet werden, dass die Kontraste mindestens dem Grad AA
der WCAG 2.2 Norm erfüllen. % TODO: ORDENTLICHE QUELLE HINZUFÜGEN!
Eine Übersicht über die Kontraste der Farben finden sie unter diesem
\href{https://contrast-grid.eightshapes.com/?version=1.1.0&background-colors=&foreground-colors=%2300005A%2C%20Dunkelblau%0D%0A%2300008C%2C%20Prim%C3%A4rblau%0D%0A%232F57B2%2C%20Blau%201%0D%0A%237369BE%2C%20Violett%201%0D%0A%23BC1589%2C%20Pink%201%0D%0A%23D20F41%2C%20Rot%201%0D%0A%23C85000%2C%20Orange%201%0D%0A%23FFC700%2C%20Gelb%201%0D%0A%23767A23%2C%20Oliv%201%0D%0A%23007D4B%2C%20Gr%C3%BCn%201%0D%0A%230A777F%2C%20Petrol%201%0D%0A%2397C6FF%2C%20Blau%202%0D%0A%23C8C8FF%2C%20Violett%202%0D%0A%23FFB9FF%2C%20Pink%202%0D%0A%23FFAAA5%2C%20Rot%202%0D%0A%23FFBE78%2C%20Orange%202%0D%0A%23FFE483%2C%20Gelb%202%0D%0A%23D2DC46%2C%20Oliv%202%0D%0A%238CE6AA%2C%20Gr%C3%BCn%202%0D%0A%238CE6D7%2C%20Petrol%202%0D%0A%23000000%2C%20Schwarz%0D%0A%23323F4B%2C%20Grau%20100%0D%0A%23566371%2C%20Grau%2080%0D%0A%237D8894%2C%20Grau%2060%0D%0A%23A5AEB8%2C%20Grau%2040%0D%0A%23D0D5DC%2C%20Grau%2020%0D%0A%23E7E9ED%2C%20Grau%2010%0D%0A%23FFFFFF%2C%20Wei%C3%9F&es-color-form__tile-size=compact&es-color-form__show-contrast=aaa&es-color-form__show-contrast=aa&es-color-form__show-contrast=aa18&es-color-form__show-contrast=dnp}{Link}.


\begin{table}[h]
  \centering
  \label{tab:colorcatalogue}
  \captionabove{Farbenkatalog der TU Dresden}
  \begin{tabular}{
    @{}
    p{\tudcdGridColumnwidth}
    @{\hspace{\tudcdGridColumnsep}}
    p{\tudcdGridColumnwidth}
    @{\hspace{\tudcdGridColumnsep}}
    p{\tudcdGridColumnwidth}
    @{\hspace{\tudcdGridColumnsep}}
    p{\tudcdGridColumnwidth}@{}}
    \colorbox{Brilliantblau}{\parbox{\tudcdGridColumnwidth}{\textcolor{Weiss}{Brilliantblau} \\}} &
    \colorbox{Blau1}{\parbox{\tudcdGridColumnwidth}{\textcolor{Weiss}{Blau1} \\ }} &
    \colorbox{Violett1}{\parbox{\tudcdGridColumnwidth}{\textcolor{Weiss}{Violett1} \\}} &
    \colorbox{Magenta1}{\parbox{\tudcdGridColumnwidth}{\textcolor{Weiss}{Magenta1} \\}} \\
    %
    \colorbox{Dunkelblau}{\parbox{\tudcdGridColumnwidth}{\textcolor{Weiss}{Dunkelblau} \\ }} &
    \colorbox{Blau2}{\parbox{\tudcdGridColumnwidth}{\textcolor{Schwarz}{Blau2} \\ }} &
    \colorbox{Violett2}{\parbox{\tudcdGridColumnwidth}{\textcolor{Schwarz}{Violett2} \\}} &
    \colorbox{Magenta2}{\parbox{\tudcdGridColumnwidth}{\textcolor{Schwarz}{Magenta2}  \\}} \\[\tudcdGridColumnsep]
    % Platzlassen
    \colorbox{Rot1}{\parbox{\tudcdGridColumnwidth}{\textcolor{Weiss}{Rot1} \\}} &
    \colorbox{Orange1}{\parbox{\tudcdGridColumnwidth}{\textcolor{Weiss}{Orange1} \\ }} &
    \colorbox{Gelb1}{\parbox{\tudcdGridColumnwidth}{\textcolor{Schwarz}{Gelb1} \\}} &
    \colorbox{Oliv1}{\parbox{\tudcdGridColumnwidth}{\textcolor{Weiss}{Oliv1} \\}} \\
    %
    \colorbox{Rot2}{\parbox{\tudcdGridColumnwidth}{\textcolor{Schwarz}{Rot2} \\ }} &
    \colorbox{Orange2}{\parbox{\tudcdGridColumnwidth}{\textcolor{Schwarz}{Orange2} \\ }} &
    \colorbox{Gelb2}{\parbox{\tudcdGridColumnwidth}{\textcolor{Schwarz}{Gelb2} \\}} &
    \colorbox{Oliv2}{\parbox{\tudcdGridColumnwidth}{\textcolor{Schwarz}{Oliv2}  \\}} \\[\tudcdGridColumnsep]
    % Platzlassen
    \colorbox{Gruen1}{\parbox{\tudcdGridColumnwidth}{\textcolor{Weiss}{Gruen1} \\}} &
    \colorbox{Tuerkis1}{\parbox{\tudcdGridColumnwidth}{\textcolor{Weiss}{Tuerkis1} \\ }} &
    \colorbox{Schwarz}{\parbox{\tudcdGridColumnwidth}{\textcolor{Weiss}{Schwarz} \\}} &
    \colorbox{Grau100}{\parbox{\tudcdGridColumnwidth}{\textcolor{Weiss}{Grau100} \\}} \\
    %
    \colorbox{Gruen2}{\parbox{\tudcdGridColumnwidth}{\textcolor{Schwarz}{Gruen2} \\ }} &
    \colorbox{Tuerkis2}{\parbox{\tudcdGridColumnwidth}{\textcolor{Schwarz}{Tuerkis2} \\ }} &
    \colorbox{Weiss}{\parbox{\tudcdGridColumnwidth}{\textcolor{Schwarz}{Weiss} \\}} &
    \colorbox{Grau80}{\parbox{\tudcdGridColumnwidth}{\textcolor{Weiss}{Grau80}  \\}} \\[\tudcdGridColumnsep]

    \colorbox{Grau60}{\parbox{\tudcdGridColumnwidth}{\textcolor{Weiss}{Grau60} \\ }} &
    \colorbox{Grau40}{\parbox{\tudcdGridColumnwidth}{\textcolor{Schwarz}{Grau40} \\ }} &
    \colorbox{Grau20}{\parbox{\tudcdGridColumnwidth}{\textcolor{Schwarz}{Grau20} \\}} &
    \colorbox{Grau10}{\parbox{\tudcdGridColumnwidth}{\textcolor{Schwarz}{Grau10}  \\}} \\
  \end{tabular}
\end{table}




\section{Die Hausschrift der TU Dresden}

\subsection{Serifen oder keine Serifen}

Die Hausschrift der TUD ist die Noto Sans, im Schriftschnitt Medium.
Da für Fließtexte eine Serifenschrift unter Umständen einen besseren
Lesefluss ermöglicht, wird die späte Option \texttt{useseriffont} bereitgestellt
\begin{lstlisting}
\documentclass{<TUDCD-Klasse>}

\TUDCDsetup{
  useseriffont
}
% oder
\TUDCDsetup{
  useseriffont=<Wahrheitswert>
}
% ...
\begin{document}
\end{lstlisting}
Diese Option schaltet auf die Serifenschrift um.
Dabei bleibt für Überschriften die serifenlose Schrift erhalten.

\subsection{Weitere Schriftschnitte}

Um alle Schriftschnitte der Noto Schriftfamilie zur Verfügung zu stellen,
werden die Makros
\begin{lstlisting}
\textcdul{Text}
\textcdel{Text}
\textcdl{Text}
\textcdsl{Text}
\textcdm{Text}
\textcdsb{Text}
\textcdb{Text}
\textcdeb{Text}
\textcdub{Text}
\end{lstlisting}
bereitgestellt. Diese wählen den jeweiligen Schriftschnitt nach der Schriftensammlung aus.

\textbf{ANMERKUNG}: Das Corporate Design lässt für Spezialanwendungen die Schriften mit
einer geringeren Laufweite (die Noto Condensed) zu. Diese wurden jedoch nicht mit eingebunden.

\TUDCDsetup{
  color={
    highlight-color=Gruen1,
    tertiary-color=Gruen2
  }
}


\chapter{Dokumente im Stil der TU Dresden}

Zum Zeitpunkt des Schreibens sind die beiden Hauptklassen
\texttt{tudcdartcl} und \texttt{tudcdreprt}.
Diese setzen das Broschürenraster der TU Dresden um.

\section{Das Broschürenraster}

Die Klassen setzen das Broschürenraster der TUD um.
Da dies in mehreren Varianten angeboten wird, werden in diesem Abschnitt
die Einstellungsmöglichkeiten dargestellt.

\subsection{Einstellung der Marginalie}

Das Broschürenraster kommt in zwei Varianten, einmal mit Marginalienspalte,
einmal ohne Marginalienspalte.
Um zwischen diesen Möglichenkeiten umzuschalten
wird die globale Option \texttt{usemargin} bereitgestellt.
\begin{lstlisting}
\documentclass[usemargin]{<Klasse>}
% Oder
\documentclass[usemargin=<Wahrheitswert>]{<Klasse>}
\end{lstlisting}
Dabei ist standardmäßig die Marginalie ausgeschaltet, die alleinige Angabe von \texttt{usemargin}
stellt diese ein.

\subsection{Einstellungen des Flattersatzes}

Gemäß Corporate Design Leitfaden werden Texte im \emph{linksbündigem Satz}
gesetzt. Dafür wird die späte Option \texttt{flush-left} bereitgestellt.
Um den linksbündigen Satz auszuschalten wird ebenfalls die späte Option \texttt{no-flush-left}
bereitgestellt.
\begin{lstlisting}
\documentclass{<Klasse>}
\TUDCDsetup{
  flush-left=<Wahrheitswert>,
}
\TUDCDsetup{
  no-flush-left
}
\begin{document}
\end{lstlisting}

\section{Schrifteinstellungen}

\subsection{Schriftgrößen}

Die Schriftgrößen, welche in den Vorlagen des Corporate Designs dargestellt sind, beziehen sich auf eine
Einheit \unit{\texpt}, welche weder \unit{\texpt\tex}, noch \unit{\texpt\word} oder \unit{\texpt\adobe} entspricht.
Daher wurde für diese Klasse die Länge \lstinline|\tudpt| definiert, welche empirisch auf \qty{73015}{\texsp} festgelegt wurde.

Zur Einstellung der Schriftgröße wird die globale Option \texttt{fontsize}
angeboten.
\begin{lstlisting}
\documentclass[fontsize=<Größe>]{<TUDCD-Klasse>}
\end{lstlisting}
Diese erlaubt folgende Werte \begin{description}
  \item[\texttt{brochure}] Hiermit wird die Schriftgröße des Haupttexts auf \qty{9}{\texpt\tud} gesetzt,
  und alle weiteren Schriftgrößenbefehle wie \lstinline|\Large|,\lstinline|\small|,\lstinline|\footnotesize| auf
  Werte gesetzt, welche dem Broschürenraster entnommen worden sind.
  Dies ist die Standardeinstellung.
  \item[$x$\texttt{tudpt}] Hiermit wird die Schriftgröße auf $x$~\unit{\texpt\tud} gesetzt, und wird intern an
  die \KOMAScript\  Option \texttt{fontsize} durchgereicht.
  In der Protokolldatei erscheint eine Information, welche Größe in \unit{\texpt\tex} an \KOMAScript\ durchgereicht wurde.
  \item[Sonst] Alle Werte welche nicht durch die oberen Einstellungen verarbeitet wurden, werden direkt an
  die \KOMAScript\  Option \texttt{fontsize} durchgereicht.
\end{description}

\section{Titelseiten}

Zum Zeitpunkt des Schreibens wird für die Konfiguration der Titelseite in \texttt{tudcdreprt},
die Umgebung \texttt{tudcdtitlepage} definiert.

\begin{lstlisting}
\begin{tudcdtitlepage}
% Inhalte
\end{tudcdtitlepage}
\end{lstlisting}

Die \lstinline|<Optionen>| stellen hierbei eine Kombination aus den allgemeinen einfachen Optionen für das Einstellen des Logos,
sowie zur Einstellung der Farben dar
\begin{lstlisting}
\TUDCDsetup{
  titlepage/highlight-color =<Farbe>,
  titlepage/background-color=<Farbe>,
  titlepage/secondary-color =<Farbe>,
  titlepage/logo-language=<de|en|auto>,
  titlepage/logo-model=<cmyk|rgb>,
  titlepage/logo-color=<white|black|blue>
}
\end{lstlisting}
Sollten die Optionen mittels \lstinline|\TUDCDsetup| gesetzt worden sein,
gelten diese ab diesem Zeitpunkt für alle nachfolgenden Umgebungen \texttt{tudcdtitlepage}

Diese können jedoch lokal überschrieben werden, indem im optionalen Argument der Umgebung
die Schlüssel neu gesetzt werden.
\begin{lstlisting}
\begin{tudcdtitlepage}[
  titlepage/highlight-color =<Farbe>,
  titlepage/background-color=<Farbe>,
  titlepage/secondary-color =<Farbe>,
  titlepage/logo-language=<de|en|auto>,
  titlepage/logo-model=<cmyk|rgb>,
  titlepage/logo-color=<white|black|blue>
]
% Inhalte
\end{tudcdtitlepage}
\end{lstlisting}
Diese Überschreibung gilt nur innerhalb dieser Umgebung.

In beiden Fällen kann das Setzen der Schlüssel mit
\begin{lstlisting}
\TUDCDsetup{
  titlepage={
    highlight-color =<Farbe>,
    background-color=<Farbe>,
    secondary-color =<Farbe>,
    logo-language=<de|en|auto>,
    logo-model=<cmyk|rgb>,
    logo-color=<white|black|blue>
  }
}
\end{lstlisting}
erfolgen.

\section{Fertige Beispieldokumente}

\subsection{Eine Artikelklasse}

Eine beispielhafte Datei für einen englischen Artikel ist angegeben.

\lstinputlisting[linerange=\%\ Start-\%\ End]{demo-article.tex}
\marginpar{Klicken Sie hier um die Datei aus der PDF zu extrahieren. \textattachfile{demo-article.tex}{\texttt{demo-article.tex}}}

\begin{figure}[hb]
  \centering
  \setlength\fboxsep{0pt}
  \fbox{\includegraphics[width=\dimexpr(3\tudcdGridColumnwidth+2\tudcdGridColumnsep)]{demo-article.pdf}}
  \caption{Resultat des der Beispieldatei}
\end{figure}

\subsection{Eine Berichtklasse im Broschürendesign}

Eine beispielhafte Datei für einen deutschsprachigen Bericht ist hier demonstriert.
Dabei wird auf die Gestaltung der Titelseite hingewiesen, welche eine andere Farbe für den Hintergrund einstellt.

\lstinputlisting[linerange=\%\ Start-\%\ End]{demo-report.tex}
\marginpar{Klicken Sie hier um die Datei aus der PDF zu extrahieren.
\textattachfile{demo-report.tex}{\texttt{demo-report.tex}}}

Um das Broschürenraster mit zwei Spalten zu setzen ist das Paket \texttt{multicol}~\cite{ctan-multicol} notwendig.

\textbf{ACHTUNG}: Die Option \texttt{twocolumn} hat bei Benutzung einen unerwünschten Nebeneffekt, wenn die
Marginalie eingeschaltet ist. Um daher Broschüren zu setzen, ist bei eingeschalteter Marginalie die Benutzung Pakets
\texttt{multicol} dringend anzuraten.

\begin{figure}[hb]
  \centering
  \setlength\fboxsep{0pt}
  \subcaptionbox{Erste Seite}{
    \fbox{\includegraphics[page=1,width=\dimexpr(2\tudcdGridColumnwidth+1\tudcdGridColumnsep)]{demo-report.pdf}}
  }
  \subcaptionbox{Zweite Seite}{
    \fbox{\includegraphics[page=2,width=\dimexpr(2\tudcdGridColumnwidth+1\tudcdGridColumnsep)]{demo-report.pdf}}
  }

  \subcaptionbox{Dritte Seite}{
    \fbox{\includegraphics[page=3,width=\dimexpr(2\tudcdGridColumnwidth+1\tudcdGridColumnsep)]{demo-report.pdf}}
  }
  \caption{Resultat des der Beispieldatei}
\end{figure}

\TUDCDsetup{
  color={
    highlight-color=Violett1,
    tertiary-color=Violett2
  }
}
\chapter{Präsentationen im Stil der TU Dresden}

Zum Zeitpunkt des Schreibens wird eine \texttt{beamer}~\cite{ctan-beamer} Präsentationsvorlage
bereitgestellt, welche jedoch die vorangegangenen Konfigurationen nicht teilt.

Dies liegt unter anderem daran, dass diese Klasse parallel zu den anderen Klassen entstanden ist,
und eine Überarbeitung noch vorliegt.

\section{Laden des Beamertemplates}

Nach erfolgter Installation des Paketbundles kann das Corporate Design der TU Dresden mittels
\begin{lstlisting}
\usetheme{tudcd}
\end{lstlisting}
eingestellt werden.

\subsection{Zusätzliche Möglichkeiten}

Die Beamerklasse definiert das zusätzliche Makro \lstinline|\secondlogos|, welches das Einbinden zusätzlicher
Zweitlogos ermöglicht.
\begin{lstlisting}
\secondlogos{Dateipfad1,Dateipfad2,Dateipfad3,...}
\end{lstlisting}
Da \lstinline|\secondlogos| eine kommaseparierte Liste an Einträgen erwartet, können die Dateipfade keine Kommata enthalten.
Die Logos werden von Links nach Rechts rechtsbündig in den dafür vorgesehenen Bereich platziert.

\section{Fertige Beispielpräsentationen}

\TUDCDsetup{
  color={
    highlight-color=Brilliantblau
  }
}
\printbibliography%



\end{document}
