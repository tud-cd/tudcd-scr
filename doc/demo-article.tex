% Start
\documentclass[logo/language=en]{tudcdartcl}
%\usepackage[T1]{fontenc}
%\usepackage[ngerman=ngerman-x-latest]{hyphsubst}

\usepackage[english]{babel}
\usepackage[babel]{microtype}
%\babelprovide[hyphenrules=ngerman-x-latest]{ngerman}
\usepackage{xcolor}
\usepackage{siunitx}
\usepackage{float}

\newcommand{\diff}{\mathop{}\!\mathrm{d}}
\title{A Unified Field Theory of Magic}
\subtitle{An internal Note}
\author{Okazaki Yumemi}
\date{2006.08.13}

\begin{document}

\maketitle%

\section{Preliminaries}


\LaTeX\, ist eine entstandene Makrosammlung für das \textbf{Textsatzsystem} \TeX\,, dass auch typographischen Laien ermöglicht einheitliche und typographisch herausfordernde Dokumente zu \textit{setzen}.
Dabei ermöglicht die strikte Trennung von Inhalt und äußerer Form den komfortablen Einbau von Automatisierungen, wie zum Beispiel das Erstellen des Inhaltsverzeichnisses, Tabellenverzeichnisses sowie des Abbildungsverzeichnisses.
Auch kann mit gewissen Schaltern (wie zum Beispiel \textbf{\textbackslash appendix}) angenehm von dem eigentlichen Dokument zu dem Anhang umgeschaltet werden.

% End

Es gibt in \LaTeX\, grundsätzlich 2 Modi: Den Textmodus, in dem dieser Text geschrieben worden ist, und den $Mathematikmodus$, und von den Mathematikmodus den Display und Inline Modus.
Mit dem Trennzeichen \$ wird der Inline-Mathematikmodus eingeführt und wieder aufgehört. Der Inline-Mathematikmodus ist grundsätzlich für das Schreiben von Variablen im Fließtext geeignet, wie zum Beispiel $S_o$, $K_{agh}$, $H_{UW}$ oder auch $e_{agh}^o$.

Der Display-Mathematikmodus ist der Modus der automatisch in der \textbf{equation} und \textbf{align} Umgebung benutzt wird, in dem unter anderem große Symbole wie das $\sum_{i=1}^n$ Summensymbol oder auch das Integralsymbol $\int_{\Omega}$ anders gesetzt werden.\begin{equation*}
    \int_{\Omega} \Phi \diff \mu = \sum_{i=1}^n \alpha_i \mu (A_i)
\end{equation*}

Um verschiedene Schriftarten zu haben, ist es dennoch sinnvoll, die einzelnen Auszeichnungen des \AmS \, Packages zu verstehen.
\begin{description}
    \item[Römische Schriftart] {\verb!\mathrm!} Hat den Effekt $\mathrm{A}$
    \item[Dicker Schriftschnitt] {\verb!\mathbf!} Hat den Effekt $\mathbf{A}$
    \item[Kalligraphischer Schriftschnitt] {\verb!\mathcal!} Hat den Effekt $\mathcal{A}$
    \item[Haupttextschriftart] {\verb!\text!} Hat den Effekt $\text{A}$
\end{description}

\subsection{Makros und deren Expansion}

Für \textbf{eine} lange Formel kann es allerdings schnell zu dem Fall kommen, dass man sich im Auszeichnen der Variablen verliert. Oft sind es bei mehreren Formeln ähnliche Symbole, die man immer erneut setzen müsste.
\LaTeX \, bietet da jedoch einen komfortablen Mechanismus der das Schreiben einzelner Teile erleichtert, nämlich die Makros. Mittels \textbackslash newcommand kann ein neues Makro deklariert werden, welches im Anschluss immer wieder verwendert werden kann. Der Aufbau ist dabei folgender:

\textbackslash newcommand \{\textbackslash makroname\}\{\textit{expandiertes Makro}\}

Hierbei meint der Begriff \textit{Expansion} den Prozess den \LaTeX \, durchläuft um das Makro \textbf{aufzulösen}. Wenn beispielsweise folgendes Makro deklariert wird:

\textbackslash newcommand \{\textbackslash WHoehe\}\{\textbackslash mathrm\{H\}\}

und dieses im Mathematikmodus geschrieben wird

\$\textbackslash WHoehe\$

tut \LaTeX \, so, als ob dort

\$\textbackslash mathrm\{H\}\$

stehen würde. Womit der Aufwand des Formelschreibens erleichtert werden kann.

\subsection{Textauszeichnungen}

Im Gegensatz zu WYSIWYG Editoren wie Word oder LibreOffice Writer, muss in \LaTeX\, Text \textit{speziell} \textbf{ausgezeichnet} \texttt{werden}, damit die jeweilige Schriftart ausgewählt werden kann.

Grundsätzlich sind \textit{schräggestellte} Schriften im Fließtext zu bevorzugen, da diese den Text als besonders auszeichnen, ohne dabei den Grauwert des Satzes unangenehm zu beeinflussen. \textbf{Fette Schriften} eignen sich vorallem in kurzen prägnanten Erläuterungstexten, welche man auf Übersichten findet, nicht aber in Technisch Wissenschaftlichen Erläuterungsberichten.

\section*{Gliederungsebenen in Latex}

\section{Ich bin die Abschnittsüberschrift}

Eine \textit{Section} zu deutsch Abschnitt ist eine Gliederungshierachie unter dem \textit{Chapter}. Diese landen auch im Inhaltsverzeichnis.

\subsection{Ich bin eine Unterabschnittsüberschrift}

Abschnitte können in Unterabschnitte gegliedert werden, welche noch im Inhaltsverzeichnis gelistet sind.

\subsubsection{Ich bin eine Unterunterabschnittüberschrift}

Die Unterunterabschnitte besitzen keine Numerierung mehr und landen auch nicht im Inhaltsverzeichnis. Sie eignen sich daher gut als Zwischenüberschriften.

\subsection*{Nicht numerierter Unterabschnitt}

\section{Einheiten}

Einheiten werden mit dem \textbf{siunitx} Package realisiert. Es hilft dabei die besonderen deutschen Typographischen Konventionen der Einheitenschreibweise zu realisieren. Es ist empfehlenswert die Dokumentation von \textbf{siunitx} selbst zu wälzen, dennoch sollen die für den Umgang wichtigsten Befehle aufgelistet werden.

\begin{description}
    \item[Zahlen] Zahlen werden mit \verb!\num! gesetzt. Dabei kann wissenschaftliche Notation verwendet werden, welche gemäß der Einstellungen in der Präambel formatiert werden.
    \num{30.0}
    \item[Einheit] Einzelstehende Einheiten werden mit \verb!\unit! gesetzt. Beispiel: \unit{\m\per\s}
    \item[Zahl mit Einheit] Einheitenbehaftete Zahlen werden mit \verb!\qty! gesetzt. Dies sorgt zusätzlich dafür, dass die Zahl von der Einheit nicht umgebrochen werden kann \qty{30}{\m\per\s}
\end{description}

Wer den Quelltext der Aufzählung vor sich stehen hat, wird feststellen, dass der Punkt durch ein Komma getauscht wurde. Des weiteren sorgt der letzte Befehl dafür, dass zwischen Zahl und Einheit kein Zeilenumbruch entstehen kann, da dies den Lesefluss, vorallem im Fließtext, stört.

\section{Formeln}

Formeln sind für den Anfänger in \LaTeX\, meist als das aufwändigste wahrgenommene Übel, allerdings argumentiere ich hierbei,
dass man in \textit{Word} sich viel länger mit den anderen Formatvorgaben des eigentlichen Textes rumschlagen muss,
anstelle sich den wirklichen Baustellen stellen zu können. Dadurch geht der Aufwand des Setzens der Gleichungen unter dem anderen Aufwand unter.

Die Trennung von Form und Inhalt funktioniert bei Formeln nicht so ganz. Über Mechanismen wie \textbackslash newcommand sollte jedoch angestrebt werden, auch diesen Teil angenehm zu gestalten.
Es gibt (vereinfacht) 2 Arten von Formeln: \textbf{equation} und \textbf{align}, sowie deren Variante mit Stern \textbf{equation*} und \textbf{align*}.

Die \textbf{equation} Umgebung eignet sich für alleinstehende Formeln, welche im Zusammenhang mit dem Fließtext stehen.
\begin{equation}
    \int_{\Omega} \Phi \diff \mu = \sum_{i=1}^n \alpha_i \mu (A_i) \label{eqn:MUSTER_Formel}
\end{equation}
Die Gleichung \ref{eqn:MUSTER_Formel} kann mit einem Label ausgestattet werden, mit dem auch Formeln referenziert werden können. Das Referenzieren kann allerdings nur in einer Umgebung ohne Stern geschehen, mit Stern fallen die Gleichungsnummern weg
\begin{equation*}
    \int_{\Omega} \Phi \diff \mu = \sum_{i=1}^n \alpha_i \mu (A_i)
\end{equation*}
Oft kommt es jedoch vor, dass mehrere Gleichungen, bzw. Ungleichungen hintereinander geschrieben werden sollen, daher ist dort \textbf{align} zu wählen. Das Kaufmannsund \& trennt hierbei die einzelnen Spalten voneinander, während \textbackslash\textbackslash\, die Zeilen voneinander trennt. \begin{align}
    x^2 &= y \\
    x   &= y
\end{align}
Jede Zeile wird dabei als eigene Gleichung angesehen. Sollte die Nummerierung in einer Zeile nicht erwünscht sein, so kann diese mit \textbackslash nonumber ausgestattet werden. \begin{align}
    x^2 &= y \\
    x   &= y \nonumber
\end{align}

In align gibt es auch die Möglichkeit viele Formeln in derselben Zeile zu schreiben \begin{align*}
    x^2 &= y & x^3 &= c\\
    x   &= y & x^3 &= c_1
\end{align*}
Dies kann auch dazu zweckentfremdet werden um Symbolische Schreibweise und Ergebnis in eine Zeile zu schreiben \begin{align}
    A_1 &= \frac{a + b}{2} \cdot h &&= \SI{30.0}{\m\squared} \\
    A_2 &= \pi \cdot r^2 &&= \SI{33.0}{\m\squared}
\end{align}


\subsection{Funktionennamen, Klammern, Brüche, Indizes und Superskripte}

Funktionennamen spezieller Funktionen, wie zum Beispiel die trigonometrischen, müssen mit \textit{aufrechten Buchstaben} geschrieben werden, dafür bieten sich die Makros \textbackslash sin, \textbackslash cos, \textbackslash tan an.
\begin{equation}
    \int_0^\pi \sin{x} \diff x = 2
\end{equation}
Dabei stehen die Funktionennamen für sich ($\sin$), können aber auch von einer geschweiften Klammer gefolgt werden $\sin{x}$, dabei wird das Argument eingerückt.

Bei großen, bzw. langen Ausdrücken bedarf es mehrerer Klammern, welche sich an die Größe des Inhalts anpassen.
\begin{equation}
    \left( \int_0^\pi \sin{x} \diff x \right)= 2
\end{equation}
Dabei wird gleichzeitig überprüft, ob die Anzahl der linken Klammern zu den der Rechten passen.
Superskripte (wie zum Beispiel Exponenten) können dabei an die Klammer gesetzt werden.
\begin{equation}
    \left(\int_5^{21} \cos^2\left(\frac{\pi x}{2}\right) - \frac{1}{2} \cos{\pi x} \diff x \right)^2 = 64
\end{equation}

Oft werden Brüche oder auch Wurzeln in der Mathematik gebraucht und geschrieben, mit \textbackslash frac und \textbackslash sqrt lassen sich diese leicht darstellen und ineinander Schachteln \begin{equation}
    c = \sqrt{\frac{g \cdot L}{2\pi} \tanh{\frac{2\pi d}{L}}}
\end{equation}

Indizes und Superskripte, sind aufgrund ihres immens Aussagefähigen Charakters gerne gesehen. Oft verstecken sich historisch gewachsene Kürzel in ihnen. In \LaTeX\, können diese leicht geschrieben werden mit dem Unterstrich \_ und dem Dach \^{} \begin{equation}
    K_i^s
\end{equation}
Sollten die Indezes bzw. Superskripte mehr als ein Zeichen beinhalten, müssen diese mit geschweiften Klammern gruppiert werden \begin{equation}
    \Delta\sigma_{c+s+r}
\end{equation}
Innerhalb des Index kann wieder ein Index angefügt werden, allerdings ist dies Typographisch sehr unschön \begin{equation}
    \varrho_{\sigma_0}
\end{equation}
\begin{equation}
    \Delta\sigma^{c+s+r}
\end{equation}
\begin{equation}
    \varrho^{\sigma^0}
\end{equation}

Diese Auszeichnungen funktionieren in allen Mathematikmodi.

\section{Tabellen und Abbildungen}

\begin{figure}[H]
    \centering
    \includegraphics[width=\columnwidth]{example-image} % anstelle von "example-image" müsste der Dateipfad dastehen.
    \caption{Ich bin die Abbildungsunterschrift}
    \label{fig:MUSTERREFERENZ}
\end{figure}

Auf die Abbildung kann ich mithilfe von \ref{fig:MUSTERREFERENZ} verweisen.

\end{document}