% Start
\documentclass[twoside,fontsize=brochure]{tudcdreprt}
\usepackage[T1]{fontenc}
\usepackage[ngerman=ngerman-x-latest]{hyphsubst}

\usepackage[ngerman]{babel}
\usepackage[babel]{microtype}
\babelprovide[hyphenrules=ngerman-x-latest]{ngerman}
\usepackage{xcolor}

\usepackage{amsmath}
\usepackage{accents}
\usepackage{mathtools}
\usepackage{csquotes}
\usepackage{siunitx}
\usepackage{multicol}
\usepackage{float}
\usepackage{tikz}

\title{Die Konstruktion des Shapes}
\subtitle{Mit \texttt{TikZ}}
\author{Yoshi Diepelt}
\date{2026}

\ExplSyntaxOn
\newcommand{\mymod}[2]{\int_mod:nn{#1}{#2}}
\ExplSyntaxOff

\TUDCDsetup{
  titlepage={
    highlight-color=Brilliantblau,
    background-color=Gelb2,
    logo-color=blue
  }
}

\begin{document}

\maketitle

\chapter{Der Shape und sein Konstruktionsschema}

\begin{multicols*}{2}
% End
Mit dem neuen Corporate Design wurde auch der \emph{Shape}
des Logos ausgetauscht. Um dennoch eine Weiterentwicklung
darzustellen wurde die Konstruktion des neuen Shapes auf
der Grundkonstruktion des alten Shapes aufgebaut.

\section{Der neue Shape}

\subsection{Konstruktion}

Zur Konstruktion des Shapes wird ein Oktagon konstruiert.

\begin{figure}[H]
  \centering
  \begin{tikzpicture}[scale=2]
    \coordinate (origin) at (0,0);
    \coordinate (top-left) at (-1,1);
    \coordinate (top-right) at (1,1);
    \coordinate (bot-left) at (-1,-1);
    \coordinate (bot-right) at (1,-1);

    \draw (origin) circle[radius=1.0];
    %
    \foreach \p in {0,1,2,...,7}{
      \fill (origin) +(22.5+\p*45:1) coordinate (oct\p) circle (1pt); %node {$\p$};
    };
    \draw (oct0) foreach \p in {1,2,3,4,5,6,7}{
      -- (oct\p)
    } -- cycle;

    \node[right] at (oct0) {$A$};
    \node[above right] at (oct1) {$B$};
    \node[above left] at (oct2) {$C$};
    \node[left] at (oct3) {$D$};
    \node[left] at (oct4) {$E$};
    \node[below left] at (oct5) {$F$};
    \node[below right] at (oct6) {$G$};
    \node[right] at (oct7) {$H$};
  \end{tikzpicture}
  \caption{Schritt 1 der Konstruktion des Shapes}
\end{figure}

Anschließend werden senkrecht zu den Seitenflächen an den Eckpunkten
sowie an den Seitenhalbierenden Geraden eingezeichnet.

\begin{figure}[H]
  \centering
  \begin{tikzpicture}[scale=2]
    \coordinate (origin) at (0,0);
    \coordinate (top-left) at (-1,1);
    \coordinate (top-right) at (1,1);
    \coordinate (bot-left) at (-1,-1);
    \coordinate (bot-right) at (1,-1);

    %\draw (origin) circle[radius=1.0];
    %

    \draw (oct0) foreach \p in {1,2,3,4,5,6,7}{
        -- (oct\p)
    } -- cycle;

    \node[below right] at (oct0) {$A$};
    \node[right] at (oct1) {$B$};
    \node[left] at (oct2) {$C$};
    \node[below left] at (oct3) {$D$};
    \node[above left] at (oct4) {$E$};
    \node[left] at (oct5) {$F$};
    \node[right] at (oct6) {$G$};
    \node[above right] at (oct7) {$H$};
    %%%%
    \foreach \p/\q in {0/1,1/2,2/3,3/4,4/5,5/6,6/7,7/0}{
      \path (oct\p) -- (oct\q) coordinate[pos=0.5] (oct\p-\q);
      \draw[Grau80] (oct\p) -- (oct\q) -- ([turn]90:2);
      \draw[Grau80] (oct\q) -- (oct\p) -- ([turn]-90:2);
      \draw[Grau80] (oct\q) -- (oct\p-\q) -- ([turn]-90:2);
      %\node at (oct\p-\q) {$\p-\q$};
    }
    \foreach \p in {0,1,2,...,7}{
      \fill (origin) +(22.5+\p*45:1) coordinate (oct\p) circle (1pt); %node {$\p$};
    };


  \end{tikzpicture}
  \caption{Schritt 2 der Konstruktion des Shapes}
  \label{fig:step2}
\end{figure}

In Abbildung~\ref{fig:step3old} wurden
die Schnittpunkte $I,\,J,\,K,\, L$ der vertikalen und horizontalen Strecken (bspw.\@ $\overline{AD}$ geschnitten mit $\overline{BG}$ ergibt Punkt $I$),
sowie die relevanten Schnittpunkte $M,\, O,\, N,\, P$ der Diagonalen eingezeichnet.

\begin{figure}[H]
  \centering
  \begin{tikzpicture}[scale=2]
    \coordinate (origin) at (0,0);
    \coordinate (top-left) at (-1,1);
    \coordinate (top-right) at (1,1);
    \coordinate (bot-left) at (-1,-1);
    \coordinate (bot-right) at (1,-1);

    %\draw (origin) circle[radius=1.0];
    %

    \draw (oct0) foreach \p in {1,2,3,4,5,6,7}{
        -- (oct\p)
    } -- cycle;

    \node[below right] at (oct0) {$A$};
    \node[right] at (oct1) {$B$};
    \node[left] at (oct2) {$C$};
    \node[below left] at (oct3) {$D$};
    \node[above left] at (oct4) {$E$};
    \node[left] at (oct5) {$F$};
    \node[right] at (oct6) {$G$};
    \node[above right] at (oct7) {$H$};
    %%%%
    \foreach \p/\q in {0/1,1/2,2/3,3/4,4/5,5/6,6/7,7/0}{
      \path (oct\p) -- (oct\q) coordinate[pos=0.5] (oct\p-\q);
      \draw[Grau80] (oct\p) -- (oct\q) -- ([turn]90:2);
      \draw[Grau80] (oct\q) -- (oct\p) -- ([turn]-90:2);
      \draw[Grau80] (oct\q) -- (oct\p-\q) -- ([turn]-90:2);
      %\node at (oct\p-\q) {$\p-\q$};
    }
    \foreach \p in {0,1,2,...,7}{
      \fill (origin) +(22.5+\p*45:1) coordinate (oct\p) circle (1pt); %node {$\p$};
    };
    %% Außeres Quadrat
    \coordinate (outerTL) at (oct0-|oct1);
    \coordinate (outerBL) at (oct1|-oct7);
    \coordinate (outerTR) at (oct0-|oct2);
    \coordinate (outerBR) at (oct7-|oct2);
    \fill (outerTL) circle (1pt) node[above right] {$I$};
    \fill (outerBL) circle (1pt) node[below left] {$J$};
    \fill (outerTR) circle (1pt) node[above left] {$K$};
    \fill (outerBR) circle (1pt) node[below right] {$L$};
    %% Inneres Quadrat
    \coordinate (innerTL) at (intersection of oct2-3--oct6-7 and oct1--oct4);
    \coordinate (innerBL) at (intersection of oct0-1--oct4-5 and oct3--oct6);
    \coordinate (innerTR) at (intersection of oct0-1--oct4-5 and oct2--oct7);
    \coordinate (innerBR) at (intersection of oct2-3--oct6-7 and oct0--oct5);
    \fill (innerTL) circle (1pt) node[below] {$M$};
    \fill (innerTR) circle (1pt) node[below] {$O$};
    \fill (innerBL) circle (1pt) node[above] {$N$};
    \fill (innerBR) circle (1pt) node[above] {$P$};
    %\coordinate (inner)
  \end{tikzpicture}
  \caption{Schritt 3 der Konstruktion des Shapes}
  \label{fig:step3old}
\end{figure}

Anschließend werden die Geraden $\overline{IM}$ und $\overline{DE}$ geschnitten,
was den Schnittpunkt $Q$ ergibt, sowie die Geraden $\overline{LP}$ und $\overline{AH}$,
womit der Schnittpunkt $R$ ermittelt wird. Diese Konstruktion ist in Abbildung \ref{fig:step4old}
dargestellt.

\begin{figure}[H]
  \centering
  \begin{tikzpicture}[scale=2]
    \coordinate (origin) at (0,0);
    \coordinate (top-left) at (-1,1);
    \coordinate (top-right) at (1,1);
    \coordinate (bot-left) at (-1,-1);
    \coordinate (bot-right) at (1,-1);

    %\draw (origin) circle[radius=1.0];
    %

    \draw (oct0) foreach \p in {1,2,3,4,5,6,7}{
        -- (oct\p)
    } -- cycle;

    \node[below right] at (oct0) {$A$};
    \node[right] at (oct1) {$B$};
    \node[left] at (oct2) {$C$};
    \node[below left] at (oct3) {$D$};
    \node[above left] at (oct4) {$E$};
    \node[left] at (oct5) {$F$};
    \node[right] at (oct6) {$G$};
    \node[above right] at (oct7) {$H$};
    %%%%
    \foreach \p/\q in {0/1,1/2,2/3,3/4,4/5,5/6,6/7,7/0}{
      \path (oct\p) -- (oct\q) coordinate[pos=0.5] (oct\p-\q);
      \draw[Grau80] (oct\p) -- (oct\q) -- ([turn]90:2);
      \draw[Grau80] (oct\q) -- (oct\p) -- ([turn]-90:2);
      \draw[Grau80] (oct\q) -- (oct\p-\q) -- ([turn]-90:2);
      %\node at (oct\p-\q) {$\p-\q$};
    }
    \foreach \p in {0,1,2,...,7}{
      \fill (origin) +(22.5+\p*45:1) coordinate (oct\p) circle (1pt); %node {$\p$};
    };
    %% Außeres Quadrat
    \coordinate (outerTL) at (oct0-|oct1);
    \coordinate (outerBL) at (oct1|-oct7);
    \coordinate (outerTR) at (oct0-|oct2);
    \coordinate (outerBR) at (oct7-|oct2);
    \fill (outerTL) circle (1pt) node[above right] {$I$}; % Falsche Bezeichnung...
    \fill (outerBL) circle (1pt) node[below left] {$J$};
    \fill (outerTR) circle (1pt) node[above left] {$K$};
    \fill (outerBR) circle (1pt) node[below right] {$L$};
    %% Inneres Quadrat
    \coordinate (innerTL) at (intersection of oct2-3--oct6-7 and oct1--oct4);
    \coordinate (innerBL) at (intersection of oct0-1--oct4-5 and oct3--oct6);
    \coordinate (innerTR) at (intersection of oct0-1--oct4-5 and oct2--oct7);
    \coordinate (innerBR) at (intersection of oct2-3--oct6-7 and oct0--oct5);
    \fill (innerTL) circle (1pt) node[below] {$M$};
    \fill (innerTR) circle (1pt) node[below] {$O$};
    \fill (innerBL) circle (1pt) node[above] {$N$};
    \fill (innerBR) circle (1pt) node[above] {$P$};
    % Schrägen
    \coordinate (outerDiag1) at (intersection of outerTL--innerTL and oct3--oct4);
    \coordinate (outerDiag2) at (intersection of outerBR--innerBR and oct0--oct7);
    \fill (outerDiag1) circle (1pt) node[left] {$Q$};
    \fill (outerDiag2) circle (1pt) node[right] {$R$};
    \draw[Grau80] (outerDiag1) -- (outerTL);
    \draw[Grau80] (outerDiag2) -- (outerBR);
    %\draw[Brilliantblau] (innerBL) -- (TShapeL) -- (oct3) -- (oct2) -- (oct1) -- (oct0) -- (TShapeR) -- (innerBR) -- cycle;
  \end{tikzpicture}
  \caption{Schritt 4 der Konstruktion des Shapes}
  \label{fig:step4old}
\end{figure}

Der Shape besteht nun aus den Polygonen mit den Ecken
$Q,\, E,\, F,\, G,\, N,\, M$ und $C,\, O, \, P,\, R,\, A,\, B$.
Die Polygone werden in Abbidung~\ref{fig:step5}, welche
zusammen den Shape ergeben.

\begin{figure}[H]
  \centering
  \begin{tikzpicture}[scale=2]
    \coordinate (origin) at (0,0);
    \coordinate (top-left) at (-1,1);
    \coordinate (top-right) at (1,1);
    \coordinate (bot-left) at (-1,-1);
    \coordinate (bot-right) at (1,-1);

    %\draw (origin) circle[radius=1.0];
    %
    \foreach \p in {0,1,2,...,7}{
      \fill (origin) +(22.5+\p*45:1) coordinate (oct\p);% circle (1pt); %node {$\p$};
    };
    %\draw (oct0) foreach \p in {1,2,3,4,5,6,7}{
    %    -- (oct\p)
    %} -- cycle;

    \node[below right] at (oct0) {$A$};
    \node[right] at (oct1) {$B$};
    \node[left] at (oct2) {$C$};
    %\node[below left] at (oct3) {$D$};
    \node[above left] at (oct4) {$E$};
    \node[left] at (oct5) {$F$};
    \node[right] at (oct6) {$G$};
    %\node[above right] at (oct7) {$H$};
    %%%%
    \foreach \p/\q in {0/1,1/2,2/3,3/4,4/5,5/6,6/7,7/0}{
      \path (oct\p) -- (oct\q) coordinate[pos=0.5] (oct\p-\q);
      \path[Grau80] (oct\p) -- (oct\q) -- ([turn]90:2);
      \path[Grau80] (oct\q) -- (oct\p) -- ([turn]-90:2);
      \path[Grau80] (oct\q) -- (oct\p-\q) -- ([turn]-90:2);
    }

    %% Außeres Quadrat
    \coordinate (outerTL) at (oct0-|oct1);
    \coordinate (outerBL) at (oct1|-oct7);
    \coordinate (outerTR) at (oct0-|oct2);
    \coordinate (outerBR) at (oct7-|oct2);
    %\node[above right] at (outerTL)  {$I$}; % Falsche Bezeichnung...
    %\node[below left]  at (outerBL) {$J$};
    %\node[above left]  at (outerTR) {$K$};
    %\node[below right] at (outerBR)  {$L$};
    %% Inneres Quadrat
    \coordinate (innerTL) at (intersection of oct2-3--oct6-7 and oct1--oct4);
    \coordinate (innerBL) at (intersection of oct0-1--oct4-5 and oct3--oct6);
    \coordinate (innerTR) at (intersection of oct0-1--oct4-5 and oct2--oct7);
    \coordinate (innerBR) at (intersection of oct2-3--oct6-7 and oct0--oct5);
    \node[above] at (innerTL)  {$M$};
    \node[below left] at (innerTR)  {$O$};
    \node[above right] at (innerBL)  {$N$};
    \node[below] at (innerBR)  {$P$};
    % Schrägen
    \coordinate (outerDiag1) at (intersection of outerTL--innerTL and oct3--oct4);
    \coordinate (outerDiag2) at (intersection of outerBR--innerBR and oct0--oct7);
    \node[left]  at (outerDiag1){$Q$};
    \node[right] at (outerDiag2) {$R$};
    \fill[Brilliantblau] % Q, E, F, G, N, M
    (outerDiag1) --
    (oct4) --
    (oct5) --
    (oct6) --
    (innerBL) --
    (innerTL) --
    cycle;
    % und C, O, P, R, A, B
    \fill[Brilliantblau]
    (outerDiag2) --
    (oct0) --
    (oct1) --
    (oct2) --
    (innerTR) --
    (innerBR) --
    cycle;
  \end{tikzpicture}
  \caption{Schritt 5 der Konstruktion des Shapes}
  \label{fig:step5}
\end{figure}

Soll der Shape die Größe $c$ besitzen, dann sind die Punkte des Shapes angegeben mit
\begin{align}
  a &= \frac{1}{2} c, & b &= \frac{1}{2} c \tan(\num{22.5}), \\
  A &= \begin{bmatrix}
    a \\
    b \\
  \end{bmatrix}, &
  B &= \begin{bmatrix}
    b \\
    a \\
  \end{bmatrix}, \\
  C &= \begin{bmatrix}
    -b \\
    a \\
  \end{bmatrix}, &
  O &= \begin{bmatrix}
    \frac{1}{2} a - \frac{1}{2}b \\
    \frac{1}{2} a - \frac{1}{2}b \\
  \end{bmatrix}, \\
  P &= \begin{bmatrix}
    \frac{1}{2} a - \frac{1}{2}b \\
    -\frac{1}{2} a + \frac{1}{2}b \\
  \end{bmatrix}, &
  R &= \begin{bmatrix}
    a \\
    -a + 2b \\
  \end{bmatrix}, \\
  E &= \begin{bmatrix}
    -a \\
    -b \\
  \end{bmatrix}, &
  F &= \begin{bmatrix}
    -b \\
    -a \\
  \end{bmatrix}, \\
  G &= \begin{bmatrix}
    b \\
    -a \\
  \end{bmatrix}, &
  N &= \begin{bmatrix}
    -\frac{1}{2} a + \frac{1}{2}b \\
    -\frac{1}{2} a + \frac{1}{2}b \\
  \end{bmatrix}, \\
  M &= \begin{bmatrix}
    -\frac{1}{2} a + \frac{1}{2}b \\
     \frac{1}{2} a - \frac{1}{2}b \\
  \end{bmatrix}, &
  Q &= \begin{bmatrix}
    -a \\
    a - 2b \\
  \end{bmatrix}.
\end{align}

\subsection{Mathematische Eigenschaften des Shapes}

Der Shape besitzt eine Punktsymmetrie um den Mittelpunkt.

Der Shape parkettiert mit den Basisvektoren
\begin{align}
  b_1 &= \begin{bmatrix}
    -\frac{3}{2}a+\frac{1}{2}b \\
    \frac{3}{2}a - \frac{5}{2}b \\
  \end{bmatrix}, &
  b_2 &= \begin{bmatrix}
    -\frac{3}{2}a + \frac{1}{2}b \\
     -\frac{1}{2}a - \frac{5}{2}b \\
  \end{bmatrix},
\end{align} die Ebene. Ein Eindruck der Parkettierung
kann in~\ref{fig:tilingtheplane} gewonnen werden.
Dies ist jedoch nicht nach den Regeln für die Verwendung
des Shapes erlaubt, weshalb von einer offiziellen Verwendung
abgesehen werden sollte.

\begin{figure}[H]
  \centering
  \begin{tikzpicture}
    \foreach \p in {0,1,2,...,5}{
      \foreach \q in {0,1,2,...,5}{

    \begin{scope}[shift={
      ({\p*(-3/2 *0.5 + 0.25*tan(22.5))+\q*(-3/2*0.5+0.25*tan(22.5))},
      {\p*(3/2*0.5-5/2*0.5*tan(22.5))+\q*(-1/2*0.5-5/2*0.5*tan(22.5))})}
    ]
    \coordinate (origin) at (0,0);

    \coordinate (A) at (0.5,{0.5*tan(22.5)});
    \coordinate (B) at ({0.5*tan(22.5)},0.5);
    \coordinate (C) at ({-0.5*tan(22.5)},0.5);
    \coordinate (O) at ({0.25-0.25*tan(22.5)},{0.25-0.25*tan(22.5)});
    \coordinate (P) at ({0.25-0.25*tan(22.5)},{-0.25+0.25*tan(22.5)});
    \coordinate (R) at (0.5,{-0.5+tan(22.5)});

    \coordinate (E) at (-0.5,{-0.5*tan(22.5)});
    \coordinate (F) at ({-0.5*tan(22.5)},-0.5);
    \coordinate (G) at ({0.5*tan(22.5)},-0.5);
    \coordinate (N) at ({-0.25+0.25*tan(22.5)},{-0.25+0.25*tan(22.5)});
    \coordinate (M) at ({-0.25+0.25*tan(22.5)},{0.25-0.25*tan(22.5)});
    \coordinate (Q) at (-0.5,{+0.5-tan(22.5)});

    \ifnum \mymod{6*\p+\q}{4}=0
    \fill[Brilliantblau] (A) -- (B) -- (C) -- (O) -- (P) -- (R);
    \fill[Brilliantblau] (E) -- (F) -- (G) -- (N) -- (M) -- (Q);
    \else
    \ifnum \mymod{6*\p+\q}{4}=1
    \fill[Dunkelblau] (A) -- (B) -- (C) -- (O) -- (P) -- (R);
    \fill[Dunkelblau] (E) -- (F) -- (G) -- (N) -- (M) -- (Q);
    \else
    \ifnum \mymod{6*\p+\q}{4}=2
    \fill[Blau1] (A) -- (B) -- (C) -- (O) -- (P) -- (R);
    \fill[Blau1] (E) -- (F) -- (G) -- (N) -- (M) -- (Q);
    \else
    \fill[Blau2] (A) -- (B) -- (C) -- (O) -- (P) -- (R);
    \fill[Blau2] (E) -- (F) -- (G) -- (N) -- (M) -- (Q);
    \fi
    \fi
    \fi
    \end{scope}
      }
    }
  \end{tikzpicture}
  \caption{Die TUD Parkettierung, hier mit 4 Farben um die einzelnen Kacheln voneinander abzugrenzen.}
  \label{fig:tilingtheplane}
\end{figure}



\end{multicols*}
\end{document}